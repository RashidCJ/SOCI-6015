% Unofficial University of Oxford Poster Template
% https://github.com/andiac/gemini-cam
% a fork of https://github.com/anishathalye/gemini

\documentclass[final]{beamer}
\usepackage[spanish, mexico]{babel}

% ====================
% Packages
% ====================

\usepackage[T1]{fontenc}
\usepackage{lmodern}
% \usepackage[size=custom,width=120,height=72,scale=1.0]{beamerposter}
\usepackage[size=custom,width=91.44,height=60.96,scale=1.0]{beamerposter} % centermeter

\usetheme{gemini}
\usecolortheme{UPRRP}
\usepackage{graphicx}
\usepackage{booktabs}
\usepackage{tikz}
\usepackage{pgfplots}
\pgfplotsset{compat=1.14}
\usepackage{anyfontsize}


% ====================
% Lengths
% ====================

% If you have N columns, choose \sepwidth and \colwidth such that
% (N+1)*\sepwidth + N*\colwidth = \paperwidth
\newlength{\sepwidth}
\newlength{\colwidth}
\setlength{\sepwidth}{0.02\paperwidth} % 5% del ancho
\setlength{\colwidth}{0.27\paperwidth} % 25% del ancho (primera y tercera columnas)
\newlength{\centralcolwidth}
\setlength{\centralcolwidth}{0.38\paperwidth} % 40% del ancho (columna central)

\newcommand{\separatorcolumn}{\begin{column}{\sepwidth}\end{column}}

% ====================
% Title
% ====================

\title{Proyecto de investigación}

\author{Rashid Marcano Rivera \inst{1} \and Yamil Ortiz Ortiz \inst{2}}

\institute[shortinst]{\inst{1} Departamento de Sociología y Antropología \samelineand \inst{2} Centro de Investigaciones Sociales}

% ====================
% Footer (optional)
% ====================

\footercontent{
  \href{https://www.example.com}{https://www.example.com} \hfill
  Conferencia ABC 2025, San Juan, Puerto Rico --- Subtítulo-1234 \hfill
  \href{mailto:juan.del.pueblo@upr.edu}{juan.del.pueblo@upr.edu}}
% (can be left out to remove footer)

% ====================
% Logo (optional)
% ====================
% Refer to https://github.com/k4rtik/uchicago-poster
% logo: https://communications.admin.ox.ac.uk/communications-resources/visual-identity/identity-guidelines/the-oxford-logo
% use this to include logos on the left and/or right side of the header:
\logoright{\includegraphics[height=7cm]{logos/Logo_FCS_UPRRP.png}}
\logoleft{\includegraphics[height=7cm]{logos/el logo.png}}

% ====================
% Body
% ====================

\begin{document}



\begin{frame}[t]
\begin{columns}[t]
\separatorcolumn

\begin{column}{\colwidth}

  \begin{block}{Título de Bloque}

    Contenido de un bloque, y texto de relleno. En esta sección se presenta el contexto de la investigación. Incluya el problema de estudio, su relevancia y los antecedentes teóricos o empíricos que fundamentan el trabajo. Puede mencionar brevemente la pregunta de investigación o hipótesis.

    %Recuerde que un póster efectivo comunica ideas de forma clara y concisa. Evite saturar los bloques con demasiado texto; priorice la legibilidad y apoye sus argumentos con elementos visuales cuando sea posible.

  \end{block}

  \begin{block}{Bloque conteniendo listado}

Aquí se presenta la evidencia previa relevante al tema de estudio. Incluya hallazgos de investigaciones anteriores que fundamenten su trabajo y establezcan qué sabemos hasta ahora sobre el fenómeno. La literatura existente ha documentado relaciones importantes entre variables socioeconómicas clave. Por ejemplo, estudios previos han encontrado asociaciones entre:

    \begin{itemize}
      \item \textbf{Desigualdad de ingreso} y políticas de empleo y capacitación laboral
      \item \textbf{Inversión pública} y movilidad social intergeneracional
      \item \textbf{Acceso a servicios} y desarrollo comunitario.
    \end{itemize}

    Estos hallazgos orientan las preguntas que guían la presente investigación.

  \end{block}

  \begin{alertblock}{Bloque enfatizado}

    Este bloque llama la atención, quizás \textbf{algo importante} debiera ir aquí.

    \begin{itemize}
      \item \textbf{Fusce dapibus tellus} vel tellus semper finibus. In
        consequat, nibh sed mattis luctus, augue diam fermentum lectus.
      \item \textbf{In euismod erat metus} non ex. Vestibulum luctus augue in
        mi condimentum, at sollicitudin lorem viverra.
    \end{itemize}

  \end{alertblock}

  \begin{block}{Nam cursus consequat egestas}

    Nulla eget sem quam. Ut aliquam volutpat nisi vestibulum convallis. Nunc a
    lectus et eros facilisis hendrerit eu non urna. Interdum et malesuada fames
    ac ante \textit{ipsum primis} in faucibus. Etiam sit amet velit eget sem
    euismod tristique. Praesent enim erat, porta vel mattis sed, pharetra sed
    ipsum. Morbi commodo condimentum massa, \textit{tempus venenatis} massa
    hendrerit quis. 

  \end{block}

\end{column}

\separatorcolumn

\begin{column}{\centralcolwidth}

  \begin{block}{Bloque con enumerado e imagen}
Quō ūsque tandem abūtere, Catilīna, patientiā nostrā? Quam diū etiam furor iste tuus nōs ēlūdet? Quem ad fīnem sēsē effrēnāta iactābit audācia?


\begin{figure}
    \centering
    \includegraphics[width=0.9\linewidth]{imágenes/Ingreso mediano distrito representativo.png}
    \caption{Pie de calce de figura.}
\end{figure}


    \begin{enumerate}
      \item \textbf{Portus Divensis (2018):} Ingressus medianus per districtum repræsentativum, secundum census publicus, variat inter regiones urbanas et rusticas. Analysis huiusmodi reddit evidentiam disparitatum œconomicorum inter municipia et districtum.
      \item \textbf{Cras vehicula blandit urna ut maximus}. Aliquam blandit nec
        massa ac sollicitudin. Curabitur cursus, metus nec imperdiet bibendum,
        velit lectus faucibus dolor, quis gravida metus mauris gravida turpis.
      \item \textbf{Vestibulum et massa diam}. Phasellus fermentum augue non
        nulla accumsan.
    \end{enumerate}
Maecenas sed porta est. Praesent mollis interdum lectus, sit amet sollicitudin risus tincidunt.
  \end{block}

  \begin{block}{Bloque con gráfico seno coseno}

    Et rutrum ex euismod vel. Pellentesque ultricies, velit in fermentum
    vestibulum, lectus nisi pretium nibh, sit amet aliquam lectus augue vel
    velit. Suspendisse rhoncus massa porttitor augue feugiat molestie. Sed
    molestie ut orci nec malesuada. Sed ultricies feugiat est fringilla
    posuere.

    \begin{figure}
      \centering
      \begin{tikzpicture}
        \begin{axis}[
            scale only axis,
            no markers,
            domain=0:2*pi,
            samples=100,
            axis lines=center,
            axis line style={-},
            ticks=none]
          \addplot[red] {sin(deg(x))};
          \addplot[blue] {cos(deg(x))};
        \end{axis}
      \end{tikzpicture}
      \caption{Pie de calce.}
    \end{figure}

  \end{block}

\end{column}

\separatorcolumn

\begin{column}{\colwidth}

  \begin{exampleblock}{Bloque enfatizado con ecuación}

    Otro tipo de bloque enfatizado.

    $$
    \int_{-\infty}^{\infty} e^{-x^2}\,dx = \sqrt{\pi}
    $$

    Interdum et malesuada fames $\{1, 4, 9, \ldots\}$ ac ante ipsum primis in
    faucibus. Cras eleifend dolor eu nulla suscipit suscipit. Sed lobortis non
    felis id vulputate.

    \heading{Encabezado dentro de bloque}

    Praesent consectetur mi $x^2 + y^2$ metus, nec vestibulum justo viverra
    nec. Proin eget nulla pretium, egestas magna aliquam, mollis neque. Vivamus
    dictum $\mathbf{u}^\intercal\mathbf{v}$ sagittis odio, vel porta erat
    congue sed. Maecenas ut dolor quis arcu auctor porttitor.

    \heading{Otra seccioncita iniciada con encabezado}

    Sed augue erat, scelerisque a purus ultricies, placerat porttitor neque.
    Donec $P(y \mid x)$ fermentum consectetur $\nabla_x P(y \mid x)$ sapien
    sagittis egestas. Duis eget leo euismod nunc viverra imperdiet nec id
    justo.

  \end{exampleblock}

  \begin{block}{Nullam vel erat at velit convallis laoreet}

    Class aptent taciti sociosqu ad litora torquent per conubia nostra, per
    inceptos himenaeos. Phasellus libero enim, gravida sed erat sit amet,
    scelerisque congue diam. Fusce dapibus dui ut augue pulvinar iaculis.

    \begin{table}
      \centering
      \begin{tabular}{l r r c}
        \toprule
        \textbf{1.\textsuperscript{\underline{a}} columna} & \textbf{2.\textsuperscript{\underline{a}} columna} & \textbf{3.\textsuperscript{\underline{a}} columna} & \textbf{4.\textsuperscript{\underline{a}} columna} \\
        \midrule
        Foo & 13.37 & 384,394 & $\alpha$ \\
        Bar & 2.17 & 1,392 & $\beta$ \\
        Baz & 3.14 & 83,742 & $\delta$ \\
        Qux & 7.59 & 974 & $\gamma$ \\
        \bottomrule
      \end{tabular}
      \caption{Leyenda de tabla.}
    \end{table}


  \end{block}

  \begin{block}{Referencias}

    \nocite{*}
    \footnotesize{\bibliographystyle{plain}\bibliography{poster}}

  \end{block}

\end{column}

\separatorcolumn
\end{columns}
\end{frame}

\end{document}
